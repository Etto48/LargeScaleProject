\chapter{Implementation}
\section{Front-End}
Our application's front end is implemented as a typical web app, with a combination of Javascript, HTML and CSS. Data is retrieved from the server through AJAX requests.
\section{Back-End}
The back-end is implemented using Java Spring Boot. Requests sent by the clients are resolved by a series of repositories which utilize Spring Data methods to handle database data.
The server can dynamically set HTML content thanks to the Thymeleaf template engine.
\section{Spring Boot}
\subsection{Spring Data Neo4j}
Spring Data Neo4j provides repository support for the Neo4j graph database. It eases development of applications with a consistent programming model that need to access Neo4j data sources.
\subsection{Spring Data MongoDB}
Spring Data for MongoDB is part of the umbrella Spring Data providing integration with the MongoDB document database, offering a familiar and consistent Spring based programming model while retaining store specific features and capabilities.
\subsubsection{Features}
\begin{itemize}
	\item \emph{MongoTemplate}: Helper class that increases productivity for common tasks.
	\item \emph{Data Repositories}: Repository interfaces including support for custom queries and aggregations.
\end{itemize}
\section{Java Entities (for MongoDB)}
\subsection{Users}
\subsubsection{Simple User - Reviewer}
\begin{lstlisting}[language = Java , frame = trBL , firstnumber = last , escapeinside={(*@}{@*)}]
@Document("users")
public class User {
    @SuppressWarnings("unused")
    private static final Logger logger = LoggerFactory.getLogger(User.class);
    @Id
    public ObjectId id;
    @Field("username")
    public String username;
    @Field("password_hash")
    public String password_hash;
    @Field("email")
    public String email;
    @Field("Top3ReviewsByLikes")
    public List<Review> top_reviews;

    public User(String username, String password_hash, String email, List<Review> top_reviews) {
        this.username = username;
        this.password_hash = password_hash;
        this.email = email;
        this.top_reviews = top_reviews;
    }

    public String getCompany_name() {
        return "";
    }
    public User() {
    }

    public User(DBObject dbObject)
    {
        this.id = (ObjectId) dbObject.get("_id");
        this.username = (String) dbObject.get("username");
        this.password_hash = (String) dbObject.get("password_hash");
        this.email = (String) dbObject.get("email");
        @SuppressWarnings("unchecked")
        List<org.bson.Document> top_review = (List<org.bson.Document>) dbObject.get("Top3ReviewsByLikes");
        this.top_reviews = top_review.stream().map(Review::new).toList();
    }

    public static User userFactory(DBObject dbObject)
    { 
        if(dbObject.get("company_name") != null)
            return new CompanyManager(dbObject);
        else if(dbObject.get("is_admin") != null && (boolean)dbObject.get("is_admin"))
            return new Admin(dbObject);
        else
            return new User(dbObject);
    }

    public String getAccountType() {
        return "User";
    }
    public String getUsername() {
        return this.username;
    }

    private String encodePassword(String password) throws NoSuchAlgorithmException {
        MessageDigest md = MessageDigest.getInstance("SHA-256");
        md.update(password.getBytes());
        byte[] digest = md.digest();
        Base64.Encoder encoder = Base64.getEncoder();
        return encoder.encodeToString(digest);
    }

    public void setPasswordHash(String password) throws NoSuchAlgorithmException{
        this.password_hash = encodePassword(password);
    }

    public boolean checkPassword(String password) throws NoSuchAlgorithmException{
        return this.password_hash.equals(encodePassword(password));
    }
}
\end{lstlisting}
\subsubsection{Company Manager}
\begin{lstlisting}[language = Java , frame = trBL , firstnumber = last , escapeinside={(*@}{@*)}]
public class CompanyManager extends User {
    @Field("company_name")
    public String company_name;
    @Override 
    public String getAccountType() {
        return "Company";
    }

    public CompanyManager() {

    }

    @Override
    public String getUsername() {
        return super.getUsername();
    }

    @Override
    public String getCompany_name() {
        return company_name;
    }

    public CompanyManager(DBObject dbObject) {
        super(dbObject);
        this.company_name = (String) dbObject.get("company_name");
    }
}
\end{lstlisting}
\subsubsection{Admin}
\begin{lstlisting}[language = Java , frame = trBL , firstnumber = last , escapeinside={(*@}{@*)}]
public class Admin extends User {
    @Override
    public String getAccountType() {
        return "Admin";
    }

    public Admin() {

    }

    public Admin(DBObject dbObject) {
        super(dbObject);
    }
}

\end{lstlisting}
\subsection{Videogames}
Documents inside the collection "videogames" can have different attributes from each other, and some may have attributes that others don't. To handle this, we decided to implement a Map object that holds all of the attributes of a videogame, whichever and however many they may be.
\begin{lstlisting}[language = Java , frame = trBL , firstnumber = last , escapeinside={(*@}{@*)}]
@Document(collection = "videogames")
public class Game {
	private static final Logger logger = LoggerFactory.getLogger(Game.class);
    @Id
    public ObjectId id;
    @Field("Name")
    public String name;
    @Field("Released")
    public String released;
    @Field("Top3ReviewsByLikes")
    public List<Review> top_reviews;


    @Field("customAttributes")
    public Map<String, Object> customAttributes = new HashMap<>();

    // other fields, getters, setters

    public Map<String, Object> getCustomAttributes() {
        return customAttributes;
    }

    public void setCustomAttributes(Map<String, Object> customAttributes) {
        this.customAttributes = customAttributes;
    }

    public void setCustomAttributes(DBObject db) {
        try {
            @SuppressWarnings("unchecked")
            Map<String,Object> map = new ObjectMapper().readValue(db.toString(), HashMap.class);
            customAttributes = map;
        }
        catch (Exception e){
            logger.error("Error while setting custom attributes: " + e.getMessage());
        }
    }

    public Game (DBObject db){
        try {
            @SuppressWarnings("unchecked")
            Map<String, Object> map = new ObjectMapper().readValue(db.toString(), HashMap.class);
            customAttributes = map;
        } catch (Exception e) {
            logger.error("Error while setting custom attributes: " + e.getMessage());
        }
        this.name = customAttributes.get("Name").toString();
        @SuppressWarnings("unchecked")
        List<org.bson.Document> reviews_object = (List<org.bson.Document>)db.get("Top3ReviewsByLikes");
        if (reviews_object != null){
            this.top_reviews = reviews_object.stream().map(Review::new).toList();
        }
        else this.top_reviews = new ArrayList<>();
        @SuppressWarnings("unchecked")
        HashMap<String,Object> released = (HashMap<String,Object>) customAttributes.get("Released");
        this.released = released.get("Release Date").toString();
    }

    public Game(String st) {
        try {
            @SuppressWarnings("unchecked")
            Map<String, Object> map = new ObjectMapper().readValue(st, HashMap.class);
            customAttributes = map;
        } catch (Exception e) {
            logger.error("Error while setting custom attributes: " + e.getMessage());
        }
        this.name = customAttributes.get("Name").toString();

        @SuppressWarnings("unchecked")
        HashMap<String, Object> released = (HashMap<String, Object>) customAttributes.get("Released");
        this.released = released.get("Release Date").toString();
    }

    public static org.bson.Document documentFromJson(String json) throws IllegalArgumentException {
        org.bson.Document doc = org.bson.Document.parse(json);
        if (doc == null)
        {
            throw new IllegalArgumentException("Invalid JSON");
        }
        if (doc.containsKey("_id")) {
            doc.remove("_id");
        }
        if (!doc.containsKey("Name")) {
            throw new IllegalArgumentException("Name is a required field");
        }
        if (!doc.containsKey("Developers")) {
            doc.put("Developers", List.of());
        }
        if (!doc.containsKey("Publishers")) {
            doc.put("Publishers", List.of());
        }
        doc.put("reviews", List.of());
        doc.put("user_review",null);
        doc.put("reviewCount",0);
        doc.put("Top3ReviewsByLikes", List.of());
        return doc;
    }
\end{lstlisting}
\subsection{Review}
\begin{lstlisting}[language = Java , frame = trBL , firstnumber = last , escapeinside={(*@}{@*)}]
public class Review {
    @Id
    public ObjectId id;
    public String game;
    public Integer score;
    public String quote;
    public String author;
    public String date;

    public Review() {
    }

    /**
     *  Set date to null if you want to use the current date
     **/
    public Review(String game, Integer score, String quote, String author, String date)
    {
        this.game = game;
        this.score = score;
        this.quote = quote;
        this.author = author;
        if (date == null)
        {
            Date d = Calendar.getInstance().getTime();
            DateFormat df = new SimpleDateFormat("yyyy-MM-dd");
            this.date = df.format(d);
        }
        else
        {
            this.date = date;
        }
    }

    public Review(DBObject dbObject)
    {
        this.id = (ObjectId) dbObject.get("_id");
        this.game = (String) dbObject.get("game");
        this.score = (Integer) dbObject.get("score");
        this.quote = (String) dbObject.get("quote");
        this.author = (String) dbObject.get("author");
        this.date = (String) dbObject.get("date");
    }

    public Review(Document document)
    {
        this.id = (ObjectId) document.get("_id");
        this.game = (String) document.get("game");
        this.score = (Integer) document.get("score");
        this.quote = (String) document.get("quote");
        this.author = (String) document.get("author");
        this.date = (String) document.get("date");
    }

    public String getId() {
        return id.toHexString();
    }
}

\end{lstlisting}
\subsection{Comment}
\begin{lstlisting}[language = Java , frame = trBL , firstnumber = last , escapeinside={(*@}{@*)}]
public class Comment {
    @Id
    public ObjectId id;
    public ObjectId reviewId;
    public String author;
    public String quote;
    public String date;

    public Comment() {}

    /**
     * Set date to null if you want to use the current date
     */
    public Comment(String reivew_id, String author, String quote, String date) {
        this.reviewId = new ObjectId(reivew_id);
        this.author = author;
        this.quote = quote;
        if (date == null)
        {
            Date d = Calendar.getInstance().getTime();
            DateFormat df = new SimpleDateFormat("yyyy-MM-dd");
            this.date = df.format(d);
        }
        else
        {
            this.date = date;
        }
    }
    
    public String getId() {
        return id.toHexString();
    }

    public String getReviewId() {
        return reviewId.toHexString();
    }
}

\end{lstlisting}
\section{Java Entities (for Neo4J)}
\subsection{Users}
\begin{lstlisting}[language = Java , frame = trBL , firstnumber = last , escapeinside={(*@}{@*)}]
@Node("User")
public class UserDTO {
    @Id
    @GeneratedValue
    public UUID id;
    @Property("username")
    public String username;
    public  UserDTO(String username){
        this.username = username;
    }
    public void setUsername(String username) {
        this.username = username;
    }
}
\end{lstlisting}
\subsection{Review}
\begin{lstlisting}[language = Java , frame = trBL , firstnumber = last , escapeinside={(*@}{@*)}]
@Node("Review")
public class ReviewDTO {

    @Id
    @GeneratedValue
    public UUID id;
    @Property("reviewId")
    public String reviewId;
    @Property("score")
    public String score;
    @Property("likeCount")
    public Integer likeCount;

    public ReviewDTO(UUID id, String score, String reviewId, Integer likeCount) {
        this.id = id;
        this.reviewId = reviewId;
        this.score = score;
        this.likeCount = likeCount;
    }

    public void setId(UUID id) {
        this.id = id;
    }

    public void setScore(String score) {
        this.score = score;
    }

    public void setReviewId(String reviewId) {
        this.reviewId = reviewId;
    }
}
\end{lstlisting}
\subsection{Game}
\begin{lstlisting}[language = Java , frame = trBL , firstnumber = last , escapeinside={(*@}{@*)}]
@Node("Game")
public class GameDTO {
    @Id
    @GeneratedValue
    public UUID id;
    @Property("name")
    public String name;

    public GameDTO(String name){
        this.name = name;
    }

    public void setName(String name){
        this.name = name;
    }
}
\end{lstlisting}