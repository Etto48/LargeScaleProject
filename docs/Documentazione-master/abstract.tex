\chapter*{Abstract}
\label{ch:abstract}

L'elaborato rientra nell'ambito di ricerca che si pone l'obiettivo di realizzare metodi lavorativi al fine di stimare grandezze fisiologiche che attualmente si misurano con metodi invasivi. Una di queste grandezze fisiologiche è la pressione arteriosa. 
%L'utilizzo dello \emph{sfigmomanometro} è un metodo che può essere utilizzato in ambito ospedaliero, ma non per tutti (parliamo di persone fragili, magari affette da particolari patologie, o semplicemente di età troppo precoce). 

%Riusciamo a classificare la pressione dividendola in scaglioni (per esempio se la massima va da x a y, invece di individuare il valore puntuale, dividiamo l'intervallo in 3 scaglioni e vediamo di classificare dove cade, con un errore di più o meno 10mm di mercurio) 
%Abbiamo trovato delle accuratezze abbastanza elevate, ma non sappiamo ancora perchè ce la fa, quali sono le variabili che aiutano la predizione? XAI. 
Tutto ciò che è basato su AI ha bisogno di dati. Quello che si può fare è un modello di classificazione della pressione. 

%Regolarmente i modelli di machine learning sono a scatola chiusa. 
Scopo finale: realizzare questo sistema, trovare la classificazione e poi magari in successiva ipotesi vedere se si riesce a  stimare il valore puntuale e con che errore. 