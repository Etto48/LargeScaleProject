\chapter{Dataset and Web Scraping}
To populate our data, we did both web scraping and random data generation.
\section{Data Scraping}
The two sources for the scraping are the following:
\begin{itemize}
	\item \textbf{MetaCritic}
	\item \textbf{MobyGames}
\end{itemize}
From the first we retrieved most of the reviews and usernames of our database.
In particular, we retrieved the score, date, author name, text, and the name of the game reviewed.
From the second one, we retrieved all of the information concerning videogames.
We kept the most important attributes, such as Name, Genre, Release Date etc. and we discarded
some MobyGames-specific attributes that were not relevant in our use case.

\section{Data Generation}

Some games that we found on MobyGames were not present on MetaCritic, so we randomly generated some
reviews and users to associate to those games.
To do so, we used Python algorithms. 
\textit{insert how algorithms work}

\begin{itemize}
	\item \textbf{Variety} : we exploited two real different sources, having different formats 
	\item \textbf{Velocity/Variability} : 
	\begin{itemize}
		\item Comments are frequently added/liked/responded to 
		\item Most active users change periodically
	\end{itemize}
\end{itemize}
\section{Raw Data}
Data was retrieved from: 
\begin{itemize}
	\item MobyGames
	\item MetaCritic 
	\item generated data 
\end{itemize}
\section{Scraping Algorithms}
We used \emph{Python} for scraping and data generation
\subsection{Videogames Scraping}
Data is scraped from \emph{MobyGames} for videogames info
\subsection{Users and Review Scraping}
Some data is scraped from \emph{Metacritic} for reviews, while other reviews are generated.
Comment on reviews are generated using algorithms 
\section{Resulting Dataset}
The final volume wanders around 400 MB, which are the result of:
\begin{itemize}
	\item 70k videogames
	\item 250k reviews 
	\item 625k comments
\end{itemize}

