%%%%%%%%%%%%%%%%%%%%%%%%%%%%%%%%%%%%%%%%%%%%%%%%%%%%%%%%%%%%%%%%%%%%%%%%%%%%%%%%
% REQUIRED PACKAGES
%%%%%%%%%%%%%%%%%%%%%%%%%%%%%%%%%%%%%%%%%%%%%%%%%%%%%%%%%%%%%%%%%%%%%%%%%%%%%%%%
\usepackage{colortbl} 
\usepackage{xcolor} 
\usepackage{diagbox}
\usepackage{a4wide}
\usepackage[T1]{fontenc}
\usepackage[utf8]{inputenc}
\usepackage[english]{babel}
\usepackage{ae,aecompl}
\usepackage{multirow}
\usepackage{ae}
\usepackage{graphicx}
\usepackage[dvipsnames]{xcolor} 
%\usepackage{bibgerm}
% \usepackage{mathptmx} %Schriftart Times New Roman
\definecolor{cadmiumgreen}{rgb}{0.0, 0.42, 0.24}
\usepackage{dsfont} 

%java
\usepackage{courier} %% Sets font for listing as Courier.
\usepackage{listings, xcolor}
\lstset{
	tabsize = 4, %% set tab space width
	showstringspaces = false, %% prevent space marking in strings, string is defined as the text that is generally printed directly to the console
	numbers = left, %% display line numbers on the left
	commentstyle = \color{green}, %% set comment color
	keywordstyle = \color{blue}, %% set keyword color
	stringstyle = \color{red}, %% set string color
	rulecolor = \color{black}, %% set frame color to avoid being affected by text color
	basicstyle = \small \ttfamily , %% set listing font and size
	breaklines = true, %% enable line breaking
	numberstyle = \tiny,
}

\usepackage{subfigure}
\usepackage{caption}
\usepackage{fancyhdr}

\usepackage{amssymb}
\usepackage{amsmath} 
\usepackage{amsfonts}

\usepackage{makeidx}
\usepackage{nomencl}
\usepackage{hyperref}
\usepackage{float}
\usepackage{color}
\usepackage[dvipsnames]{xcolor}
\usepackage{listings} %zur Einbindung von anderen Codes

\usepackage{braket}
\usepackage{url}

\usepackage{tikz}
\usepackage{amsmath,amsthm,mathrsfs,amssymb,dsfont,frontespizio,enumerate,bm}
\usetikzlibrary{positioning}

% ImeMINE
\newcommand{\facciatabianca}{\newpage\shipout\null\stepcounter{page}}

\usepackage{amsmath}
\newcommand{\SubItem}[1]{
	{\setlength\itemindent{15pt} \item[-] #1}
}
\DeclareMathOperator*{\argmax}{arg\,max}
\DeclareMathOperator*{\argmin}{arg\,min}


\usepackage[style=alphabetic,backend=bibtex,natbib=true]{biblatex} 

\addbibresource{biblio.bib}

\usepackage{matlab-prettifier}
\lstset{basicstyle=\ttfamily\footnotesize,breaklines=true}

% \usepackage[style=chicago-authordate,natbib=true,backend=biber]{biblatex}

%MINE THIS CHAP FOR CODE 
% Copyright 2009--2017 by Olivier Verdier
% License: see the file LICENSE.rst
\NeedsTeXFormat{LaTeX2e}
\ProvidesPackage{pythonhighlight}[2017/02/09 python code highlighting; provided by Olivier Verdier <olivier.verdier@gmail.com>]


\RequirePackage{listings}
\RequirePackage{xcolor}

\renewcommand*{\lstlistlistingname}{Code Listings}
\renewcommand*{\lstlistingname}{Code Listing}
\definecolor{gray}{gray}{0.5}
\colorlet{commentcolour}{green!50!black}

\colorlet{stringcolour}{red!60!black}
\colorlet{keywordcolour}{magenta!90!black}
\colorlet{exceptioncolour}{yellow!50!red}
\colorlet{commandcolour}{blue!60!black}
\colorlet{numpycolour}{blue!60!green}
\colorlet{literatecolour}{magenta!90!black}
\colorlet{promptcolour}{green!50!black}
\colorlet{specmethodcolour}{violet}

\newcommand*{\framemargin}{3ex}

\newcommand*{\literatecolour}{\textcolor{literatecolour}}

\newcommand*{\pythonprompt}{\textcolor{promptcolour}{{>}{>}{>}}}

\lstdefinestyle{mypython}{
	%\lstset{
		%keepspaces=true,
		language=python,
		showtabs=true,
		tab=,
		tabsize=2,
		basicstyle=\ttfamily\footnotesize,%\setstretch{.5},
		stringstyle=\color{stringcolour},
		showstringspaces=false,
		alsoletter={1234567890},
		otherkeywords={\%, \}, \{, \&, \|},
		keywordstyle=\color{keywordcolour}\bfseries,
		emph={and,break,class,continue,def,yield,del,elif ,else,%
			except,exec,finally,for,from,global,if,import,in,%
			lambda,not,or,pass,print,raise,return,try,while,assert,with},
		emphstyle=\color{blue}\bfseries,
		emph={[2]True, False, None},
		emphstyle=[2]\color{keywordcolour},
		emph={[3]object,type,isinstance,copy,deepcopy,zip,enumerate,reversed,list,set,len,dict,tuple,xrange,append,execfile,real,imag,reduce,str,repr},
		emphstyle=[3]\color{commandcolour},
		emph={Exception,NameError,IndexError,SyntaxError,TypeError,ValueError,OverflowError,ZeroDivisionError},
		emphstyle=\color{exceptioncolour}\bfseries,
		%upquote=true,
		morecomment=[s]{"""}{"""},
		commentstyle=\color{commentcolour}\slshape,
		%emph={[4]1, 2, 3, 4, 5, 6, 7, 8, 9, 0},
		emph={[4]ode, fsolve, sqrt, exp, sin, cos,arctan, arctan2, arccos, pi,  array, norm, solve, dot, arange, isscalar, max, sum, flatten, shape, reshape, find, any, all, abs, plot, linspace, legend, quad, polyval,polyfit, hstack, concatenate,vstack,column_stack,empty,zeros,ones,rand,vander,grid,pcolor,eig,eigs,eigvals,svd,qr,tan,det,logspace,roll,min,mean,cumsum,cumprod,diff,vectorize,lstsq,cla,eye,xlabel,ylabel,squeeze},
		emphstyle=[4]\color{numpycolour},
		emph={[5]__init__,__add__,__mul__,__div__,__sub__,__call__,__getitem__,__setitem__,__eq__,__ne__,__nonzero__,__rmul__,__radd__,__repr__,__str__,__get__,__truediv__,__pow__,__name__,__future__,__all__},
		emphstyle=[5]\color{specmethodcolour},
		emph={[6]assert,yield},
		emphstyle=[6]\color{keywordcolour}\bfseries,
		emph={[7]range},
		emphstyle={[7]\color{keywordcolour}\bfseries},
		% emph={[7]self},
		% emphstyle=[7]\bfseries,
		literate=*%
		{:}{{\literatecolour:}}{1}%
		{=}{{\literatecolour=}}{1}%
		{-}{{\literatecolour-}}{1}%
		{+}{{\literatecolour+}}{1}%
		{*}{{\literatecolour*}}{1}%
		{**}{{\literatecolour{**}}}2%
		{/}{{\literatecolour/}}{1}%
		{//}{{\literatecolour{//}}}2%
		{!}{{\literatecolour!}}{1}%
		%{(}{{\literatecolour(}}{1}%
		%{)}{{\literatecolour)}}{1}%
		{[}{{\literatecolour[}}{1}%
		{]}{{\literatecolour]}}{1}%
		{<}{{\literatecolour<}}{1}%
		{>}{{\literatecolour>}}{1}%
		{>>>}{\pythonprompt}{3}%
		,%
		%aboveskip=.5ex,
		frame=trbl,
		%frameround=tttt,
		%framesep=.3ex,
		rulecolor=\color{black!40},
		%framexleftmargin=\framemargin,
		%framextopmargin=.1ex,
		%framexbottommargin=.1ex,
		%framexrightmargin=\framemargin,
		%framexleftmargin=1mm, framextopmargin=1mm, frame=shadowbox, rulesepcolor=\color{blue},#1
		%frame=tb,
		backgroundcolor=\color{white},
		breakindent=.5\textwidth,frame=single,breaklines=true%
		%}
}

\newcommand*{\inputpython}[3]{\lstinputlisting[firstline=#2,lastline=#3,firstnumber=#2,frame=single,breakindent=.5\textwidth,frame=single,breaklines=true,style=mypython]{#1}}

\lstnewenvironment{python}[1][]{\lstset{style=mypython}}{}

\lstdefinestyle{mypythoninline}{
	style=mypython,%
	basicstyle=\ttfamily,%
	keywordstyle=\color{keywordcolour},%
	emphstyle={[7]\color{keywordcolour}},%
	emphstyle=\color{exceptioncolour},%
	literate=*%
	{:}{{\literatecolour:}}{2}%
	{=}{{\literatecolour=}}{2}%
	{-}{{\literatecolour-}}{2}%
	{+}{{\literatecolour+}}{2}%
	{*}{{\literatecolour*}}2%
	{**}{{\literatecolour{**}}}3%
	{/}{{\literatecolour/}}{2}%
	{//}{{\literatecolour{//}}}{2}%
	{!}{{\literatecolour!}}{2}%
	%{(}{{\literatecolour(}}{2}%
	%{)}{{\literatecolour)}}{2}%
	{[}{{\literatecolour[}}{2}%
	{]}{{\literatecolour]}}{2}%
	{<}{{\literatecolour<}}{2}%
	{<=}{{\literatecolour{<=}}}3%
	{>}{{\literatecolour>}}{2}%
	{>=}{{\literatecolour{>=}}}3%
	{==}{{\literatecolour{==}}}3%
	{!=}{{\literatecolour{!=}}}3%
	{+=}{{\literatecolour{+=}}}3%
	{-=}{{\literatecolour{-=}}}3%
	{*=}{{\literatecolour{*=}}}3%
	{/=}{{\literatecolour{/=}}}3%
	%% emphstyle=\color{blue},%
}

\newcommand*{\pyth}{\lstinline[style=mypythoninline]}


%END 